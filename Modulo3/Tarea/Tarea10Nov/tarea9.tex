\documentclass[12pt]{article}

% ------------------------------
% PAQUETES BÁSICOS
% ------------------------------
\usepackage[utf8]{inputenc}
\usepackage[T1]{fontenc}
\usepackage[spanish, es-nodecimaldot]{babel} % <- punto decimal
\decimalpoint
\usepackage{amsmath, amssymb, amsfonts}
\usepackage{graphicx}
\usepackage{geometry}
\usepackage{fancyhdr}
\usepackage[
    backend=biber,
    style=numeric,
    sorting=none
]{biblatex}
\addbibresource{biblio.bib}
\usepackage{caption}
\usepackage{subcaption}
\usepackage{enumitem}
\usepackage{hyperref}
\usepackage{xcolor}
\usepackage{booktabs} % tablas bonitas

% ------------------------------
% CONFIGURACIÓN GENERAL
% ------------------------------
\geometry{letterpaper, margin=2.5cm}
\setlength{\parskip}{0.5em}
\setlength{\parindent}{0pt}

% Cambiar "Cuadro" por "Tabla"
\addto\captionsspanish{\renewcommand{\tablename}{Tabla}}

% ------------------------------
% ENCABEZADO Y PIE DE PÁGINA
% ------------------------------
\pagestyle{fancy}
\fancyhf{}
\chead{\textbf{Técnicas Estadísticas y Minería de Datos}}
%\lhead{\textbf{TEyMD}}
%\rhead{\textbf{Tarea 5}}


% Pie de página con línea superior
\renewcommand{\footrulewidth}{0.4pt} % grosor de la línea (0 para quitarla)
\rfoot{\thepage}
\lfoot{\textbf{Harold Vázquez Corrilo}}
% ------------------------------
% COLORES Y LINKS
% ------------------------------
\hypersetup{
    colorlinks=true,
    linkcolor=blue!60!black,
    urlcolor=blue!60!black,
    citecolor=green!60!black
}

% ------------------------------
% INICIO DEL DOCUMENTO
% ------------------------------
\begin{document}

%\begin{center}
 %   {\LARGE \textbf{Tarea 1}}\\[4pt]
  %  {\large Curso: Nombre del curso}\\[2pt]
   % {\large Profesor: Nombre del profesor}\\[2pt]
    %{\large Fecha: \today}
%\end{center}

%\hrule
%\vspace{1em}
\newcommand{\m}[1]{\pmb{#1}} % comando para vectores y matrices en negrita


%%%%%%
%% Tema: Ji cuadrada
%%%%%%

\subsection*{JI CUADRADA}

Sean $Y_1, Y_2, \ldots, Y_n$ variables aleatorias independientes e idénticamente distribuidas con distribución normal estándar $N(0,1)$. 
Entonces la variable aleatoria definida como \cite{weisstein_chi_distribution}
\begin{align}
X = \sum_{i=1}^n Y_i^2
\label{eq:chi_squared}
\end{align}
tiene una distribución Ji-cuadrada con $n$ grados de libertad, es decir, $X \sim \chi^2(n)$.



%%%%%%%%%%%%%%%%%%%
%%%%%%%% Tema : T-STUDENT 
%%%%%%%%%%%%%%%%%%%

\subsection*{T-STUDENT}

Sea $X$ una variable aleatoria con distribución normal estándar $N(0,1)$ y sea $Y$ una variable aleatoria independiente con distribución $\chi^2(n)$. 
Entonces la variable aleatoria definida como \cite{weisstein_t_distribution}
\begin{align}
T = \frac{X}{\sqrt{Y/n}}
\label{eq:t_student}
\end{align}
tiene una distribución t-Student con $n$ grados de libertad, es decir, $T \sim t(n)$.

%%%%%%%%%%%%%%%%%%%
%%%%%%%% Tema : F-FISHER
%%%%%%%%%%%%%%%%%%%

\subsection*{F-FISHER}
Sean $Y_n$ y $Y_m$ variables aleatorias independientes con distribuciones $\chi^2(n)$ y $\chi^2(m)$ respectivamente.
Se define el estadístico $F_{n,m}$ como el radio de dispersión de las dos distribuciones Ji-cuadrada \cite{weisstein_f_distribution}, es decir, 
\begin{align}
F_{n,m} \equiv \frac{(Y_n/n)}{(Y_m/m)}.
\label{eq:f_fisher}
\end{align}



\printbibliography

% ------------------------------
% FIN DEL DOCUMENTO
% ------------------------------
\end{document}