\documentclass[12pt]{article}

% ------------------------------
% PAQUETES BÁSICOS
% ------------------------------
\usepackage[utf8]{inputenc}
\usepackage[T1]{fontenc}
\usepackage[spanish, es-nodecimaldot]{babel} % <- punto decimal
\decimalpoint
\usepackage{amsmath, amssymb, amsfonts}
\usepackage{graphicx}
\usepackage{geometry}
\usepackage{fancyhdr}
\usepackage[
    backend=biber,
    style=numeric,
    sorting=none
]{biblatex}
\addbibresource{biblio.bib}
\usepackage{caption}
\usepackage{subcaption}
\usepackage{multirow}
\usepackage{enumitem}
\usepackage{hyperref}
\usepackage{xcolor}
\usepackage{booktabs} % tablas bonitas

% ------------------------------
% CONFIGURACIÓN GENERAL
% ------------------------------
\geometry{letterpaper, margin=2.5cm}
\setlength{\parskip}{0.5em}
\setlength{\parindent}{0pt}

% Cambiar "Cuadro" por "Tabla"
\addto\captionsspanish{\renewcommand{\tablename}{Tabla}}

% ------------------------------
% ENCABEZADO Y PIE DE PÁGINA
% ------------------------------
\pagestyle{fancy}
\fancyhf{}
\chead{\textbf{Técnicas Estadísticas y Minería de Datos}}
%\lhead{\textbf{TEyMD}}
%\rhead{\textbf{Tarea 5}}


% Pie de página con línea superior
\renewcommand{\footrulewidth}{0.4pt} % grosor de la línea (0 para quitarla)
\rfoot{\thepage}
\lfoot{\textbf{Harold Vázquez Corrilo}}
% ------------------------------
% COLORES Y LINKS
% ------------------------------
\hypersetup{
    colorlinks=true,
    linkcolor=blue!60!black,
    urlcolor=blue!60!black,
    citecolor=green!60!black
}

% ------------------------------
% INICIO DEL DOCUMENTO
% ------------------------------
\begin{document}

%\begin{center}
 %   {\LARGE \textbf{Tarea 1}}\\[4pt]
  %  {\large Curso: Nombre del curso}\\[2pt]
   % {\large Profesor: Nombre del profesor}\\[2pt]
    %{\large Fecha: \today}
%\end{center}

%\hrule
%\vspace{1em}
\newcommand{\m}[1]{\pmb{#1}} % comando para vectores y matrices en negrita


%%%%%%
%% Tema: Ji cuadrada
%%%%%%

\subsection*{Forma funcional lineal}

Se utilizaron las variables de la data \textbf{greene11\_3} en tasas de crecimiento para estimar el siguiente modelo de regresión lineal:
\begin{center}

    Modelo 1: MCO, usando las observaciones 1951--1985 ($T$ = 35)\\
    Variable dependiente: pc\_C\\
    
    \vspace{1em}
    
    \begin{tabular}{lr@{.}lr@{.}lr@{.}lr@{.}l}
      &
     \multicolumn{2}{c}{Coeficiente} &
      \multicolumn{2}{c}{Desv.\ Típica} &
       \multicolumn{2}{c}{Estadístico $t$} &
        \multicolumn{2}{c}{valor p} \\[1ex]
    const &
      0&917379 &
        0&376027 &
          2&440 &
            0&0202 \\
    pc\_Y &
      0&722665 &
        0&0985844 &
          7&330 &
          2&05e-08 \\
    \end{tabular}
    
    \vspace{1ex}
    \begin{tabular}{lrlr}
    Media de la vble. dep. &  3.349682 & D.T. de la vble. dep. &  1.671687 \\
    Suma de cuad. residuos &  36.15000 & D.T. de la regresión &  1.046640 \\
    $R^2$ &  0.619531 & $R^2$ corregido &  0.608001 \\
    $F(1, 33)$ &  53.73502 & Valor p (de $F$) &  2.05\textrm{e--08} \\
    Log-verosimilitud & $-$50.22860 & Criterio de Akaike &  104.4572 \\
    Criterio de Schwarz &  107.5679 & Hannan--Quinn &  105.5310 \\
    $\hat{\rho}$ & $-$0.106674 & Durbin--Watson &  2.124016 \\
    \end{tabular}
\end{center}
al cual se le realizaron pruebas de RESET de Ramsey para detectar posibles errores de especificación en la forma funcional del modelo.
Se obtuvieron los siguientes resultados:
\begin{table}[h!]
    \centering
    \begin{tabular}{lcc}
        \toprule
        Prueba & Valor p & Rechaza $H_0$ al $95\%$\\
        \midrule
        Cuadrados & 0.0182409 & Sí \\
        Cubos & 0.0186227 & Sí \\
        \bottomrule
    \end{tabular}
    \caption{Valores p de las pruebas RESET de Ramsey para el Modelo 1.} 
    \label{tab:reset_mod1}
\end{table}
Con base en los resultados de la tabla \ref{tab:reset_mod1}, se rechaza la hipótesis nula 
de especificación adecuada. Para la corrección de este supuesto se agregaron 
las variables pc\_Y$^2$ y pc\_Y$^3$ al modelo, obteniéndose el siguiente resultado:
    
\begin{table}[h!]
    \centering
    \begin{tabular}{lcclcc}
        \toprule
        Variable agregada & $R^2$ & $R^2$ corregido & Prueba & Valor p & Rechaza $H_0$ al $95\%$\\
        \midrule
        \multirow{2}{3em}{sq\_pc\_Y} & \multirow{2}{3em}{0.681204} & \multirow{2}{3em}{0.661279} & Cuadrados & 0.78708 & No \\
        & & & Cubos & 0.981524 & No \\
        \hline
        \multirow{2}{3em}{cubo\_pc\_Y} & \multirow{2}{3em}{0.679652} & \multirow{2}{3em}{0.659630} & Cuadrados & 0.652204 & No \\
        & & & Cubos & 0.527084 & No \\
        \bottomrule
    \end{tabular}
    \caption{Modelos con variables agregadas\\ y resultados de las pruebas RESET de Ramsey.} 
    \label{tab:reset_cuad_cub}
\end{table}
Además de agregar las variables cuadráticas y cúbicas, se utilizó el método 
de mínimos cuadrados no lineales para estimar el siguiente modelo:
\begin{center}

    Modelo 6: MC. no lineales, usando las observaciones 1951--1985 ($T$ = 35)\\
    \verb!pc_C = alfa + beta * sgnY * (absY^gamma)!\\
    
    \vspace{1em}
    
    \begin{tabular}{lr@{.}lr@{.}lr@{.}lr@{.}l}
      &
     \multicolumn{2}{c}{Estimación} &
      \multicolumn{2}{c}{Desv.\ Típica} &
       \multicolumn{2}{c}{Estadístico $t$} &
        \multicolumn{2}{c}{valor p} \\[1ex]
    alfa &
      0&303938 &
        0&449547 &
          0&6761 &
            0&5038 \\
    $\beta$ &
      1&49164 &
        0&401658 &
          3&714 &
            0&0008 \\
    $\gamma$ &
      0&634669 &
        0&135090 &
          4&698 &
          4&77e-05 \\
    \end{tabular}
    
    \vspace{1ex}
    \begin{tabular}{lrlr}
    Media de la vble. dep. &  3.349682 & D.T. de la vble. dep. &  1.671687 \\
    Suma de cuad. residuos &  31.44268 & D.T. de la regresión &  0.991254 \\
    $R^2$ &  0.669074 & $R^2$ corregido &  0.648391 \\
    Log-verosimilitud & $-$47.78717 & Criterio de Akaike &  101.5743 \\
    Criterio de Schwarz &  106.2404 & Hannan--Quinn &  103.1851 \\
    $\hat{\rho}$ & $-$0.145914 & Durbin--Watson &  2.191558 \\
    \end{tabular}
    \end{center}
donde sgnY = sign(pc\_Y) y absY = |pc\_Y|, ya que pc\_Y toma valores negativos, lo 
cual no permite elevarla a una potencia fraccionaria directamente.\\
Después de obtener tres modelos que corrigen el supuesto de forma funcional, notamos que 
cada uno es una posible solución al problema, pero por simplicidad y por capacidad 
explicativa se opta por el Modelo 2 (con variable cuadrática) como la mejor opción.


%%%%%%%%%%%%%%%%%%%
%%%%%%%% Tema : Cambio estructural
%%%%%%%%%%%%%%%%%%%

\subsection*{Cambio estructural}
Se utilizaron las variables de la data \textbf{data9-2} para estimar el siguiente modelo de regresión lineal:
\begin{center}

    Modelo 1: MCO, usando las observaciones 1961--1995 ($T$ = 35)\\
    Variable dependiente: pc\_D\\
    
    \vspace{1em}
    
    \begin{tabular}{lr@{.}lr@{.}lr@{.}lr@{.}l}
      &
     \multicolumn{2}{c}{Coeficiente} &
      \multicolumn{2}{c}{Desv.\ Típica} &
       \multicolumn{2}{c}{Estadístico $t$} &
        \multicolumn{2}{c}{valor p} \\[1ex]
    const &
      21&5575 &
        48&9344 &
          0&4405 &
            0&6624 \\
    pc\_r &
      $-$1&23871 &
        2&43345 &
          $-$0&5090 &
            0&6141 \\
    \end{tabular}
    
    \vspace{1ex}
    \begin{tabular}{lrlr}
    Media de la vble. dep. &  17.77867 & D.T. de la vble. dep. &  283.0141 \\
    Suma de cuad. residuos &   2702081 & D.T. de la regresión &  286.1490 \\
    $R^2$ &  0.007791 & $R^2$ corregido & -0.022276 \\
    $F(1, 33)$ &  0.259118 & Valor p (de $F$) &  0.614114 \\
    Log-verosimilitud & $-$246.6111 & Criterio de Akaike &  497.2222 \\
    Criterio de Schwarz &  500.3329 & Hannan--Quinn &  498.2960 \\
    $\hat{\rho}$ & $-$0.106771 & Durbin--Watson &  1.644262 \\
    \end{tabular}
\end{center}
Al cual se le aplicó la prueba RV de Quandt para detectar posibles cambios estructurales 
en los parámetros del modelo. Se obtuvo el siguiente gráfico:\\
\begin{figure}[h!]
    \centering
    \includegraphics[width=0.55\textwidth]{imagenes/rv_quandt.pdf}
    \caption{Resultados prueba RV de Quandt.}
    \label{fig:rv_quandt}
\end{figure}\\
En la figura \ref{fig:rv_quandt} se observa que el valor del estadístico de la prueba es 
mayor que el valor crítico al $5\%$ en varios puntos antes del año 1973. También
se obtuvo un valor $$p_{\text{asintótico}} = 2.27933e-06,$$ por lo que se rechaza la hipótesis nula 
de que no hay cambio estructural.\\
También se realizó la prueba de Chow para un posible cambio estructural en 1971, obteniéndose:
$$ p = 0.0018241.$$ Chow también rechaza la hipótesis nula de que no hay cambio estructural en 1971.\\ 
Por último se realizaron las pruebas CUSUM y CUSUMSQ, obteniéndose los siguientes gráficos:
\begin{figure}[h!]
    \centering
    \begin{subfigure}[b]{0.47\textwidth}
        \includegraphics[width=0.9\textwidth]{imagenes/cusum.pdf}
        \caption{Prueba CUSUM.}
        \label{fig:cusum}
    \end{subfigure}
    \hfill
    \begin{subfigure}[b]{0.47\textwidth}
        \includegraphics[width=0.9\textwidth]{imagenes/cusumsq.pdf}
        \caption{Prueba CUSUMSQ.}
        \label{fig:cusumsq}
    \end{subfigure}
    \caption{Gráficos de las pruebas CUSUM y CUSUMSQ.}
    \label{fig:cusum_cusumsq}
\end{figure}\\
La figura \ref{fig:cusumsq} muestra que la serie sale de las bandas de confianza, 
por lo tanto hay evidencia de cambio estructural según la prueba CUSUMSQ.\\
Dado que tres de las cuatro pruebas realizadas indican la presencia de un cambio estructural,
se concluye que el modelo presenta un cambio estructural en sus parámetros durante el periodo analizado.\\
Para corregir este supuesto, se restringió la muestra, el periodo analizado fue de 1976
a 1975, y se obtuvo el siguiente modelo:\\
\begin{center}

    Modelo 2: MCO, usando las observaciones 1976--1995 ($T$ = 20)\\
    Variable dependiente: pc\_D\\
    
    \vspace{1em}
    
    \begin{tabular}{lr@{.}lr@{.}lr@{.}lr@{.}l}
      &
     \multicolumn{2}{c}{Coeficiente} &
      \multicolumn{2}{c}{Desv.\ Típica} &
       \multicolumn{2}{c}{Estadístico $t$} &
        \multicolumn{2}{c}{valor p} \\[1ex]
    const &
      10&8798 &
        7&13391 &
          1&525 &
            0&1446 \\
    pc\_r &
      $-$0&491531 &
        0&326291 &
          $-$1&506 &
            0&1493 \\
    \end{tabular}
    
    \vspace{1ex}
    \begin{tabular}{lrlr}
    Media de la vble. dep. &  10.16971 & D.T. de la vble. dep. &  32.88025 \\
    Suma de cuad. residuos &  18241.38 & D.T. de la regresión &  31.83410 \\
    $R^2$ &  0.111957 & $R^2$ corregido &  0.062622 \\
    $F(1, 18)$ &  2.269297 & Valor p (de $F$) &  0.149308 \\
    Log-verosimilitud & $-$96.53593 & Criterio de Akaike &  197.0719 \\
    Criterio de Schwarz &  199.0633 & Hannan--Quinn &  197.4606 \\
    $\hat{\rho}$ & $-$0.020072 & Durbin--Watson &  2.010282 \\
    \end{tabular}
    \end{center}
Lo primero que observamos es que a pesar de que pc\_r no es estadísticamente significativa,
el valor de $R^2$ y $R^2$ corregido mejoró con respecto al modelo 1. Además, al aplicar las pruebas
de cambio estructural al modelo 2, se obtuvieron los siguientes resultados:
\begin{figure}[h!]
    \centering
    \includegraphics[width=0.55\textwidth]{imagenes/rv_quandt2.pdf}
    \caption{Resultados prueba RV de Quandt.}
    \label{fig:rv_quandt2}
\end{figure}\\
En este caso el valor del estadístico de la prueba es menor que el valor crítico al $5\%$ en todos los puntos,
y se obtuvo un valor $$p_{\text{asintótico}} = 0.38144,$$ por lo que no se rechaza la hipótesis nula. El valor 
máximo de la prueba se encuentra en 1984. Se realizó la prueba de Chow para un posible cambio estructural en 1984, obteniéndose:
$$ p = 0.0718637.$$ Chow no rechaza la hipótesis nula de que no hay cambio estructural en 1984.\\
Por último se realizaron las pruebas CUSUM y CUSUMSQ, obteniéndose los siguientes gráficos:\vspace{2em}
\begin{figure}[h!]
    \centering
    \begin{subfigure}[b]{0.47\textwidth}
        \includegraphics[width=0.9\textwidth]{imagenes/cusum2.pdf}
        \caption{Prueba CUSUM.}
        \label{fig:cusum2}
    \end{subfigure}
    \hfill
    \begin{subfigure}[b]{0.47\textwidth}
        \includegraphics[width=0.9\textwidth]{imagenes/cusumsq2.pdf}
        \caption{Prueba CUSUMSQ.}
        \label{fig:cusumsq2}
    \end{subfigure}
    \caption{Gráficos de las pruebas CUSUM y CUSUMSQ.}
    \label{fig:cusum_cusumsq2}
\end{figure}\\
La prueba CUSUM (figura \ref{fig:cusum2}) muestra que la serie se mantiene dentro de las bandas de confianza, pero la figura \ref{fig:cusumsq2} muestra que la serie sale de las bandas de confianza. 
Sin embargo, dado que la mayoría de las pruebas no indican la presencia de un cambio estructural,
se concluye que el modelo 2 no presenta un cambio estructural en sus parámetros durante el periodo analizado.\\
Por lo que se considera que se ha corregido el supuesto de cambio estructural en el modelo original.

%\printbibliography

% ------------------------------
% FIN DEL DOCUMENTO
% ------------------------------
\end{document}