\documentclass[12pt]{article}

% ------------------------------
% PAQUETES BÁSICOS
% ------------------------------
\usepackage[utf8]{inputenc}
\usepackage[T1]{fontenc}
\usepackage[spanish, es-nodecimaldot]{babel} % <- punto decimal
\decimalpoint
\usepackage{amsmath, amssymb, amsfonts}
\usepackage{graphicx}
\usepackage{geometry}
\usepackage{fancyhdr}
\usepackage{enumitem}
\usepackage{hyperref}
\usepackage{xcolor}
\usepackage{booktabs} % tablas bonitas

% ------------------------------
% CONFIGURACIÓN GENERAL
% ------------------------------
\geometry{letterpaper, margin=2.5cm}
\setlength{\parskip}{0.5em}
\setlength{\parindent}{0pt}

% Cambiar "Cuadro" por "Tabla"
\addto\captionsspanish{\renewcommand{\tablename}{Tabla}}

% ------------------------------
% ENCABEZADO Y PIE DE PÁGINA
% ------------------------------
\pagestyle{fancy}
\fancyhf{}
\chead{\textbf{Técnicas Estadísticas y Minería de Datos}}


% Pie de página con línea superior
\renewcommand{\footrulewidth}{0.4pt} % grosor de la línea (0 para quitarla)
\rfoot{\thepage}
\lfoot{\textbf{Harold Vázquez Corrilo}}
% ------------------------------
% COLORES Y LINKS
% ------------------------------
\hypersetup{
    colorlinks=true,
    linkcolor=blue!60!black,
    urlcolor=blue!60!black,
    citecolor=blue!60!black
}

% ------------------------------
% INICIO DEL DOCUMENTO
% ------------------------------
\begin{document}

%\begin{center}
 %   {\LARGE \textbf{Tarea 1}}\\[4pt]
  %  {\large Curso: Nombre del curso}\\[2pt]
   % {\large Profesor: Nombre del profesor}\\[2pt]
    %{\large Fecha: \today}
%\end{center}

%\hrule
%\vspace{1em}

\subsection*{Omisión de variables relevantes}

Para la prueba de omisión de variables relevantes se considero el modelo de 
regresión lineal múltiple con $pc\_G$ como variable dependiente 
y $pc\_Pg$, $pc\_Y$, $pc\_Pnc$, $pc\_Puc$, $pc\_Ppt$, $pc\_Pd$, $pc\_Pn$,
 $pc\_Ps$ y $pc\_Pop$ como variables independientes. Después se llevó a cabo
 eliminación secuencial hasta llegar al modelo reducido, las variables 
 eliminadas fueron: $pc\_Pnc$, $pc\_Ppt$, $pc\_Pn$, $pc\_Ps$ y $pc\_Pop$.\\
 A las variables eliminadas se les hizo una prueba $F$ de significancia 
 conjunta bajo la hipótesis nula $H_0$: los parámetros son cero para las
 variables. El resultado de la prueba fue un valor $$p=0.225012,$$ considerando
 una confianza del $95\%$ no podemos rechazar la hipótesis nula.\\
 Además de la prueba de significancia conjunta, también tenemos los criterios
 de información de Akaike, Hannan-Quinn y Schwarz, con los siguientes valores:

 \begin{table}[h]
    \centering
    \begin{tabular}{|c|c|c|}
        \hline
        \hline
        \textbf{Criterio} & \textbf{Modelo Completo} & \textbf{Modelo Reducido} \\
        \hline
        Akaike & 141.8180 & 141.0091 \\
        Hannan-Quinn & 147.1870 & 143.6936 \\
        Schwartz & 157.3714 & 148.7858 \\
        \hline
        \hline
    \end{tabular}
    \caption{Criterios de información para ambos modelos}
 \end{table}

Notamos que en los tres casos los valores son menores para el modelo reducido,
lo que indica que el modelo reducido es mejor que el modelo completo.\\
Por lo tanto, con base en la prueba de significancia conjunta realizada a
las variables eliminadas y en los criterios de información, concluimos
que en el modelo reducido no se omiten variables relevantes.

\subsection*{Adición de variables irrelevantes}
Para la prueba de adición de variables irrelevantes se consideró el modelo
reducido que consta de $pc\_G$ como variable dependiente y $pc\_Pg$, $pc\_Y$,
$pc\_Puc$ y $pc_Pd$ como variables independientes. Después se llevó a cabo
la adición de las variables $pc\_Pnc$, $pc\_Ppt$, $pc\_Pn$, $pc\_Ps$ y $pc\_Pop$,
a las cuales se les hizo una prueba $F$ de significancia conjunta. EL resultado
de la prueba fue un valor $$p=0.225012,$$ considerando una confianza del $95\%$
no podemos rechazar la hipótesis nula $H_0$: los parámetros son cero para las
variables añadidas.\\
Además de la prueba de significancia conjunta, gretl menciona que al añadir
las variables, no mejora ninguno de los criterios de información
(Akaike, Hannan-Quinn y Schwarz).\\
Con base en la prueba de significancia conjunta y en los criterios de información,
concluimos que el mejor modelo es al que no se le añaden variables.

% ------------------------------
% FIN DEL DOCUMENTO
% ------------------------------
\end{document}