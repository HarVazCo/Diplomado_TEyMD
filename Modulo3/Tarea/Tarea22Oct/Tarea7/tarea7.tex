\documentclass[12pt]{article}

% ------------------------------
% PAQUETES BÁSICOS
% ------------------------------
\usepackage[utf8]{inputenc}
\usepackage[T1]{fontenc}
\usepackage[spanish, es-nodecimaldot]{babel} % <- punto decimal
\decimalpoint
\usepackage{amsmath, amssymb, amsfonts}
\usepackage{graphicx}
\usepackage{geometry}
\usepackage{fancyhdr}
\usepackage[
    backend=biber,
    style=numeric,
    sorting=none
]{biblatex}
%\addbibresource{biblio.bib}

\usepackage{enumitem}
\usepackage{hyperref}
\usepackage{xcolor}
\usepackage{booktabs} % tablas bonitas

% ------------------------------
% CONFIGURACIÓN GENERAL
% ------------------------------
\geometry{letterpaper, margin=2.5cm}
\setlength{\parskip}{0.5em}
\setlength{\parindent}{0pt}

% Cambiar "Cuadro" por "Tabla"
\addto\captionsspanish{\renewcommand{\tablename}{Tabla}}

% ------------------------------
% ENCABEZADO Y PIE DE PÁGINA
% ------------------------------
\pagestyle{fancy}
\fancyhf{}
\chead{\textbf{Técnicas Estadísticas y Minería de Datos}}
%\lhead{\textbf{TEyMD}}
%\rhead{\textbf{Tarea 5}}


% Pie de página con línea superior
\renewcommand{\footrulewidth}{0.4pt} % grosor de la línea (0 para quitarla)
\rfoot{\thepage}
\lfoot{\textbf{Harold Vázquez Corrilo}}
% ------------------------------
% COLORES Y LINKS
% ------------------------------
\hypersetup{
    colorlinks=true,
    linkcolor=blue!60!black,
    urlcolor=blue!60!black,
    citecolor=green!60!black
}

% ------------------------------
% INICIO DEL DOCUMENTO
% ------------------------------
\begin{document}

%\begin{center}
 %   {\LARGE \textbf{Tarea 1}}\\[4pt]
  %  {\large Curso: Nombre del curso}\\[2pt]
   % {\large Profesor: Nombre del profesor}\\[2pt]
    %{\large Fecha: \today}
%\end{center}

%\hrule
%\vspace{1em}


%%%%%%
%% Tema: Pruebas de raiz unitaria
%%%%%%

\subsection*{Pruebas de raíz unitaria}
\begin{itemize}
    \item \textbf{Prueba aumentada de Dickey-Fuller (ADF):} \\
    Esta prueba consiste en hacer una regresión de la primera diferencia 
    de la serie contra su valor rezagado y rezagos de la primera diferencia, 
    opcionalmente incluyendo una constante y una tendencia. La expresión 
    general es:
    \begin{align}
        \Delta y_t = \mu + \beta_0 t + \delta y_{t-1} + \sum_{i=1}^{p} \beta_i \Delta y_{t-i} + \varepsilon_t
    \end{align}
    donde $\Delta$ es el operador de diferencia, $y_t$ es la serie temporal,
    $t$ es una variable de tendencia, $p$ es el número de rezagos incluidos,
    y $\varepsilon_t$ es el término de error. La prueba de raíz unitaria es 
    sobre el valor de $\delta$. La hipótesis nula ($H_0$) es que la serie tiene una raíz
    unitaria, es decir, $\delta=0$.\\
    Se recomienda correr la ecuación de prueba con diferentes valores de $p$,
    añadiendo o quitando la constante y la tendencia según sea necesario. Además,
    de repetir la prueba sobre la primera diferencia de la serie para identificar
    si la serie es integrada de orden uno $I(1)$ o de un orden mayor.\\
    Esta prueba opera bajo el supuesto de que los errores son ruido blanco.

    \item \textbf{Prueba de Phillips-Perron (PP):} \\
    Esta prueba no exige que se cumplan los supuestos de no autocorrelación y
    homocedasticidad de los errores. La expresión de la prueba es la siguiente:
    \begin{align}
        \Delta y_t = \mu + \beta T + \delta y_{t-1} + \varepsilon_t
    \end{align}
    donde $\Delta$ es el operador de diferencia, $y_t$ es la serie temporal,
    $T$ el número de observaciones, y $\varepsilon_t$ es el término de error.\\
    Los estadísticos de esta prueba se calculan de forma no paramétrica. La hipótesis nula ($H_0$) es que la serie tiene una raíz
    unitaria, es decir, $\delta=0$.\\
    Tanto ADF como PP son pruebas que tienen como hipótesis alternativa 
    que la serie es estacionaria con media cero.

    \item \textbf{Diferencias entre las pruebas ADF y PP:} \\
    La diferencia que parece ser más relevante entre ambas pruebas es que la prueba PP
    no requiere que los errores sean ruido blanco, mientras que la prueba ADF sí lo hace.
    Además, la prueba ADF incluye términos de rezago de la primera diferencia, 
    mientras que la prueba PP tiene una expresión más simple. Esto puede llevar
    a que la prueba PP sea más rápida de calcular en ciertos casos.\\
    Por último, los estadísticos de la prueba ADF se calculan de forma paramétrica, mientras que
    los de la prueba PP se calculan de forma no paramétrica.\\

    \item \textbf{Cambios estructurales:}\\
    Es importante identificar la presencia de cambios estructurales cuando 
    se aplican pruebas de raíz unitaria, ya que las pruebas tradicionales como ADF y PP
    tienen poco poder para identificar si la trayectoria es estacionaria o no. 
    En presencia de cambios estructurales hay un sesgo hacia el no rechazo 
    de la hipótesis nula de raíz unitaria.

    \item \textbf{Conclusión sobre las variables analizadas:}\\
    A las doce variables analizadas se les aplicaron las pruebas de DF, DFA, PP y KPSS.
    Las conclusiones de estas pruebas es que todas las series son no estacionarias en niveles. 
    Además, también se les realizó la prueba de Phillips-Perron que considera el cambio
    estructural, y todos resultados fueron consistentes con las pruebas ya realizadas, excepto
    la cuenta corriente que resultó ser $I(0)$ bajo esta prueba.
    
\end{itemize}



%\printbibliography

% ------------------------------
% FIN DEL DOCUMENTO
% ------------------------------
\end{document}